\documentclass[10pt]{beamer}

\usetheme[subsectionpage=progressbar]{metropolis}
\usepackage{appendixnumberbeamer}

\usepackage{ucs}
\usepackage[utf8x]{inputenc}

\usepackage{booktabs}
\usepackage[scale=2]{ccicons}

\usepackage{pgfplots}
\usepgfplotslibrary{dateplot}

\usepackage{xspace}
\usepackage{dsfont}
\usepackage{graphicx}
\newcommand{\themename}{\textbf{\textsc{metropolis}}\xspace}

\title{Fallstudien II}
%\subtitle{A modern beamer theme}
\date{12. Dezember 2020}
\author{Laura Kampmann, Christian Peters, Alina Stammen}
%\institute{TU Dortmund}
%\titlegraphic{\hfill\includegraphics[height=1.5cm]{logo.pdf}}

\begin{document}

\maketitle

\begin{frame}{Inhalt}
  \setbeamertemplate{section in toc}[sections numbered]
  \tableofcontents
\end{frame}

%\section{Einleitung}

\begin{frame}{Einleitung}
    Hier stehen ein paar Dinge \"uber die Einleitung:
    \begin{itemize}
        \item Dies
        \item und
        \item das
    \end{itemize}
\end{frame}




\section{TaskI}
\begin{frame}{frame}
hallo 
\end{frame}


\subsection{Extreme Gradient Boosting}

\begin{frame}{Extreme Gradient Boosting}
    Intro
\end{frame}

\begin{frame}{Features}
    Features
\end{frame}

\begin{frame}{Tuning}
    Tuning
\end{frame}

\begin{frame}{Validierung}
	\begin{figure}[h]
		\centering
		\includegraphics[scale=0.33]{kreuzvalidierung}
		\caption{Einteilungen in Trainings- und Testdatensätze bei der Kreuzvalidierung für Zeitreihen.}
		\label{kreuzvalidierung}
	\end{figure}
\end{frame}

\begin{frame}{Out-of-Sample Vorhersagen}
    \begin{figure}[h]
        \centering
        \includegraphics[scale=0.33]{plots/xgboost/uplink/scatter_colored_axes_fixed}
        \caption{XGBoost Out-of-Sample Vorhersagen der Upload-Rate}
        \label{xgboost_scatter_colored_uplink}
    \end{figure}
\end{frame}

\begin{frame}{Out-of-Sample Vorhersagen}
    \begin{figure}[h]
        \centering
        \includegraphics[scale=0.33]{plots/xgboost/downlink/scatter_colored_axes_fixed}
        \caption{XGBoost Out-of-Sample Vorhersagen der Download-Rate}
        \label{xgboost_scatter_colored_downlink}
    \end{figure}
\end{frame}

\subsection{Regression mit ARMA-Fehlern}
\nocite{forecasting}

\begin{frame}{Situation}
	\begin{figure}[]
		%\centering
		\includegraphics[scale=0.25]{highway_drive_one}
		\caption{Grafik der auf der ersten Testfahrt im Szenario ``Highway'' gemessenen Datenübertragungsrate.}
		\label{highway_drive_one}
	\end{figure}
	
	\begin{itemize}
		\item Zeitreihe $y_1,...,y_n$ (Zielvariable)
		\item k Zeitreihen $x_{i,1},...,x_{i,n}$ für $i=1,...,k$ (Einflussvariablen)
	\end{itemize}
\end{frame}

\begin{frame}{Regression mit ARMA-Fehlern}
	\textbf{Lineares Regressionsmodell} \\
	$$y_{t} = c + \beta_1 x_{1,t} + ... + \beta_k x_{k,t} + \epsilon_t \text{ mit Fehler } \epsilon_t \text{ und Konstante }c$$ \\
	Annahmen an Fehler:
	\begin{itemize}
		\item $E((\epsilon_1,...,\epsilon_n)^T) = 0$
		\item $Cov((\epsilon_1,...,\epsilon_n)^T)=\sigma^2 \mathds{1}_n$
	\end{itemize}
	\textbf{Annahmen sind in unserer Situation nicht einhaltbar!}
\end{frame}

\begin{frame}{Regression mit ARMA-Fehlern}
	\textbf{ARMA(p, q)}: Zusammengesetzes Modell aus\\
	\begin{itemize}
		\item $AR(p)$ (Auto Regressive): Linearkombination aus
		\begin{itemize}
			\item p vorherige Beobachtungen, 
			\item Konstante
			\item Fehler
		\end{itemize}
		\item $MA(q)$ (Moving Average): Linearkombination aus
		\begin{itemize}
			\item q vorherige Fehler
			\item Konstante
			\item aktueller Fehler
		\end{itemize}
	\end{itemize}
\end{frame}

\begin{frame}{Regression mit ARMA-Fehlern}
	\textbf{Anwendung auf Regressionsfehler} \\
	\underline{Erinnerung}: Fehler $(\epsilon_1,...,\epsilon_n)$ des linearen Modells sind autokorreliert $\Rightarrow$ erfüllen Voraussetzungen nicht\\
	\underline{Lösung}: Wende ARMA-Modell auf Fehler an \\
	\textbf{Modellgleichung Regression mit ARMA-Fehlern}:
	$$y_t = c + \sum_{i=1}^{k}{\beta_i x_{i,t}} + \underbrace{\sum_{j=1}^{p}{\phi_j\epsilon_{t-j}}}_{\text{vergangene Fehler LM}} + \underbrace{\sum_{k=1}^{q}{\theta_k e_{t-k}}}_{\text{vergangene Fehler ARMA}} + e_t$$
	
\end{frame}

%\begin{frame}{Regression mit ARMA-Fehlern}
%	\textbf{h-Schritt Punktvorhersage}\\
%	\begin{itemize}
%		\item Ersetze Beobachtungen zu zukünftigen Zeitpunkten mit deren Vorhersagen
%		\item Ersetze Fehler an vergangenen Zeitpunkten durch das entsprechende Residuum
%		\item Ersetze Fehler an zukünftigen Zeitpunkten durch 0
%	\end{itemize}
%	\underline{Beispiel}: $h=2, k=1, p=2, q=2$
%	\begin{alignat*}{2}
%		y_t &= c + \beta_1 x_{t} + \epsilon_t \text{ mit }&&\epsilon_{t} = \phi_1\epsilon_{t-1} + \phi_2\epsilon_{t-2} + \theta_1e_{t-1} + \theta_2e_{t-2} + e_t \\		
%		\widehat{y_{t+1}} &= c + \beta_1 x_{t} + \widehat{\epsilon_{t+1}} \text{ mit }&& \widehat{\epsilon_{t+1}} = \phi_1\epsilon_{t} + \phi_2\epsilon_{t-1} + \theta_1e_{t} + \theta_2e_{t-1} + \underbrace{\widehat{e_{t+1}}}_{=0}\\
%		\widehat{y_{t+2}} &= c + \beta_1 x_{t} + \widehat{\epsilon_{t+2}}\text{ mit }&&\widehat{\epsilon_{t+2}} = \phi_1\widehat{\epsilon_{t+1}} + \phi_2\epsilon_t + \theta \underbrace{\widehat{e_{t+1}}}_{=0} + \theta e_t + \underbrace{\widehat{e_{t+2}}}_{=0}
%	\end{alignat*}
%\end{frame}

	




\subsection{Validierung}
\begin{frame}{frame}
hallo 
\end{frame}

\section{TaskII}
\begin{frame}{frame}
hallo 
\end{frame}

\subsection{DatentransformationTaskII}
\begin{frame}{frame}
hallo 
\end{frame}



%\begin{frame}[standout]
%  Irgendwas zum Schluss
%\end{frame}

\appendix

\begin{frame}[allowframebreaks]{Literatur}

  \bibliography{literatur}
  \bibliographystyle{abbrv}

\end{frame}

\begin{frame}{Regression mit ARMA-Fehlern}
	\textbf{h-Schritt Punktvorhersage}\\
	\begin{itemize}
		\item Ersetze Beobachtungen zu zukünftigen Zeitpunkten mit deren Vorhersagen
		\item Ersetze Fehler an vergangenen Zeitpunkten durch das entsprechende Residuum
		\item Ersetze Fehler an zukünftigen Zeitpunkten durch 0
	\end{itemize}
	\underline{Beispiel}: $h=2, k=1, p=2, q=2$
	\begin{alignat*}{2}
		y_t &= c + \beta_1 x_{t} + \epsilon_t \text{ mit }&&\epsilon_{t} = \phi_1\epsilon_{t-1} + \phi_2\epsilon_{t-2} + \theta_1e_{t-1} + \theta_2e_{t-2} + e_t \\		
		\widehat{y_{t+1}} &= c + \beta_1 x_{t} + \widehat{\epsilon_{t+1}} \text{ mit }&& \widehat{\epsilon_{t+1}} = \phi_1\epsilon_{t} + \phi_2\epsilon_{t-1} + \theta_1e_{t} + \theta_2e_{t-1} + \underbrace{\widehat{e_{t+1}}}_{=0}\\
		\widehat{y_{t+2}} &= c + \beta_1 x_{t} + \widehat{\epsilon_{t+2}}\text{ mit }&&\widehat{\epsilon_{t+2}} = \phi_1\widehat{\epsilon_{t+1}} + \phi_2\epsilon_t + \theta \underbrace{\widehat{e_{t+1}}}_{=0} + \theta e_t + \underbrace{\widehat{e_{t+2}}}_{=0}
	\end{alignat*}
\end{frame}

\end{document}
