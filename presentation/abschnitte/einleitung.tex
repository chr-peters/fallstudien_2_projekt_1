\section{Einleitung}

\begin{frame}{Einleitung}
	\begin{itemize}
		\item $\textbf{Motivation}$: Verbesserung mobiler Kommunikation von Endgäten \\
		$\rightarrow$ Vermeiden von z.B. "packet loss" und als Folge auch Retransmission
		\item Wie kann das erreicht werden? \\
		$\rightarrow$ Datenratenprädiktion um optimalen Zeitpunkt zum Senden von Daten zu ermitteln
    \end{itemize}
\end{frame}


\begin{frame}{Datenbeschreibung}
Situation:
	\begin{itemize}
		\item echt Welt Messungen im öffentlichen LTE Netzwerk der 3 deutschen Mobilfunkanbieter o2, T-Mobile und Vodafone
		\item Aufteilung in mehrere Szenarien: "campus", "urban", "suburban" und "highway"
		\item pro Mobilfunkanbieter und Szenario wurden 10 Testfahrten durchgeführt
	\end{itemize}
\end{frame}


\begin{frame}{Datenbeschreibung}
	\begin{itemize}
		\item "context": passive Messungen 1s\\
		$\rightarrow$ $\textbf{RSRP}$ \\
		$\rightarrow$ $\textbf{RSRQ}$ \\
		$\rightarrow$ $\textbf{CQI}$ \\
		$\rightarrow$ $\textbf{TA}$ \\
		$\rightarrow$ $\textbf{velocity}$ \\
		$\rightarrow$ $\textbf{Cell ID}$  \\
		$\rightarrow$ $\textbf{payload size}$ \\
		\item "ul" / "dl": aktive Messungen 10s \\
		$\rightarrow$ $\textbf{throughput}$ - Datenrate \\
		\item "cells":
		$\rightarrow$ RSRP / RSRQ der Nachbarzellen
	\end{itemize}
\end{frame}



