\begin{frame}{Gradient Boosted Trees}
    \begin{itemize}
        \item Kann man aus vielen "schwachen" Lernern einen starken Lerner konstruieren?
            \begin{itemize}
                \item[$\Rightarrow$] Ja, Boosting ist eines der m\"achtigsten Konzepte des Machine Learning \cite{elements}
            \end{itemize}
        \item Kombination von einfachen CART B\"aumen zu einem starken Ensemble
            \begin{itemize}
                \item[$\Rightarrow$] \"Ahnlich zu Random Forest
            \end{itemize}
        \item Der Unterschied zum Random Forest liegt im Training!
    \end{itemize}
\end{frame}

\begin{frame}{Training von Gradient Boosted Trees}
    \begin{itemize}
        \item B\"aume werden nacheinander zum Ensemble hinzugef\"ugt
        \item Jeder neue Baum versucht, die Schw\"achen seiner Vorg\"anger "auszub\"ugeln"
            \begin{itemize}
                \item[$\Rightarrow$] \textit{Additives Training}
            \end{itemize}
        \item Je mehr B\"aume aufgenommen werden, desto geringer wird der Training-Error (das Modell wird aber komplexer)
            \begin{itemize}
                \item[$\Rightarrow$] Kontrolle des \textit{Bias-Variance Tradeoffs}
                \item[$\Rightarrow$] Zus\"atzlich gibt es Regularisierungs-Parameter
            \end{itemize}
    \end{itemize}
\end{frame}

\begin{frame}{Implementierung: XGBoost}
    \begin{itemize}
        \item Liefert state-of-the-art Performance in einer Vielzahl von ML-Problemen
        \item In 2015 haben 19/25 Gewinner von Kaggle-Competitions XGBoost eingesetzt
        \item Kann problemlos auf mehrere Milliarden Training Samples skaliert werden 
        \item L\"asst sich aber auch hervorragend auf ressourcenbegrenzten Systemen einsetzen \cite{XGBoost}
    \end{itemize}
\end{frame}