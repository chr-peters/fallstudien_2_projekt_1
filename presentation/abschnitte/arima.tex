\subsection{ARIMA}


\begin{frame}{Aufgabenstellung Task I: Vorhersage der Datenrate}
\begin{itemize}
\item weitere Ansätze zur Vorhersage der Zielgröße "throughput"
\item Aussagekraft der Einflussvariablen
\end{itemize}
\end{frame}

\begin{frame}{Lösungsansätze}
\begin{itemize}
\item XGboost
\item ARMA Modell mit Regressionsfehlern
\end{itemize}
\end{frame}

\begin{frame}{ARMA Modell mit Regressionsfehlern}
Lineares Modell: $y = \beta \cdot X + \epsilon$, wobei $\epsilon$ Störfaktor und $\epsilon_{i}$ i.i.d.
\begin{itemize}
\item Problem: Autokorrelation
\item Lösung: Anwendung des ARMA Modells auf die Regressionsfehler
\end{itemize}
\end{frame}

\begin{frame}{ARMA Modell mit Regressionsfehlern}
$ARMA(p,q)$: zusammengesetzes Modell aus
\begin{itemize}
\item $AR(p)$ (Auto Regressive): basiert auf vergangenen Werten $\epsilon_{i}$ des Response 
\item $MA(q)$ (Moving Average): basiert auf Fehlern $e_{i}$ zwischen vergangenen Vorhersagen und wahrem Wert des Response
\item Modellgleichung des ARMA Modells: 
\begin{center}
$\epsilon_{i} = \underbrace{\phi_{1}\epsilon_{i-1}+ ... + \phi_{p}\epsilon_{i-p}}_{AR(p)} \underbrace{- \theta_{1}e_{i-1} - ... - \theta_{q}e_{i-q}}_{MA(q)} + \eta_{i}$,
\end{center}
mit $\eta_{i}$ als Störfaktor
\end{itemize}
\end{frame}

\begin{frame}{Wahl der Parameter p, q}
\begin{itemize}
\item ACF (Autocorrelationfunktion) und PACF (partial Autocorrelationfunction) beschreiben die Korrelation der Lags mit dem aktuellen Zeitpunkt
\item PACF beinhaltet nur direkte Einflüsse 
\item ACF dagegen betrachtet auch solche Einflüsse die indirekt sind
\item die Funktionen legen damit die Wahl der Parameter p und q der Modell fest
\end{itemize}
\begin{center}
hier könnte ein Bild sein
\end{center}
\end{frame}

\begin{frame}{Modellgleichung ARMA mit Regressionsfehlern}
Insgesamt ist die Modellgleichung gegeben durch
\begin{center}
$y_{i} = \beta_{0} + \beta_{1}x_{1} + ... + \beta_{p}x_{p} + \phi_{1}\epsilon_{i-1}+ ... + \phi_{p}\epsilon_{i-p}$ \\
$- \theta_{1}e_{i-1} - ... - \theta_{q}e_{i-q}  + \eta_{i}$
\end{center}
\end{frame}
